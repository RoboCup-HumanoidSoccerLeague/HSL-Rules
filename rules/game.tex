% !TeX root = ../Rules.tex
% !TeX spellcheck = en_US
\section{Game Process}
\label{sec:game}



Law 7 and the following ones.
\subsection{Law 7 – The Duration of the Match}
% Following subsection was edited by Prof. Azer Babaev 
\subsection{Law 8 – The Start and Restart of Play}
\label{sec:Law 8 – The Start and Restart of Play}

A kick-off starts both halves of a match, both halves of extra time and restarts play after a goal has been scored. Free kicks (direct or indirect), penalty kicks,
throw-ins, a goal kicks and corner kicks are other restarts (see Laws 13–17).
A dropped ball is the restart when the referee stops play and the Law does not
require one of the above restarts.

If an offence occurs when the ball is not in play, this does not change how
play is restarted.

\subsubsection{Kick-off}
\textbf{Procedure}
\begin{itemize}
    \item The referee tosses a coin and the team that wins the toss decides which goal to attack in the first half or to take the kick-off.

    \item Depending on the above, their opponents take the kick-off or decide which goal to attack in the first half.

    \item The team that decided which goal to attack in the first half takes the kick-off to start the second half.

    \item For the second half, the teams change ends and attack the opposite goals. The kick-off at the beginning of the second half has to be taken by the team which did not take the kick-off in the beginning of the first half. 

    \item After a team scores a goal, a kick-off has to be taken by their opponents.
\end{itemize}

For every kick-off:
\begin{itemize}
    \item all players must be in their own half of the field of play,

    \item the opponents of the team taking the kick-off must be outside from the center circle until ball is in play,

    \item the ball must be stationary on the center mark,

    \item the referee gives a signal,

    \item the ball is in play when it is kicked and clearly moves,

    \item if the ball does not move 10 seconds after the referee has given the signal then the ball is in play and the opponents of the team taking the kick-off may enter into the center circle, \info{From humanoid league, to be discussed}

    \item a goal may not be scored directly against the opponents from the kick-off,

    \item if the team taking the kick-off has three or more robots on the field then two different robots need to touch the ball before scoring a goal. If a team taking the kick-off has only two or less robots on the field, the robot taking the kick-off has to touch the ball at least one time outside the center circle before scoring a goal.\info{This statement needs confirmation by TC} 
\end{itemize}
\textbf{Offences and sanctions}

In the event of any kick-off procedure offence, the kick-off is retaken.

\subsubsection{Dropped ball}\info{Dropped ball definition and procedure are different in FIFA rules and in RC-HL rules. RC-HL rules are used in current edition because FIFA rules of dropped ball are hardly to be implemented by robots}
\textbf{Definition of dropped ball}

A dropped ball is a method of restarting play when, while the ball is still in play, the referee is required to stop play
temporarily for any reason not mentioned elsewhere in the Laws of the Game. In the virtual competition, the only reason for a dropped ball to be called is that the ball has moved less than 5 centimeters in the last 2 minutes of play.

\textbf{Procedure}

The game is continued at the center mark. A goal can be scored directly from a dropped ball. The procedure for dropped ball is the same as for kick-off, except that the players of both teams must be outside the center circle. The ball is in play immediately after the referee gives the signal

\textbf{Offences and sanctions}
\begin{itemize}
    \item If a player moves too close to the ball before the referee gives the signal, a kick-off is awarded to the opponent team.

    \item The ball is dropped again:
        \begin{itemize}
            \item if it is touched by a player before it makes contact with the ground,
            \item if the ball leaves the field of play after it makes contact with the ground, without a player touching it.
        \end{itemize}
\end{itemize}
% End of subsection which was edited by Prof. Azer Babaev

\subsection{Law 9 – The Ball In and Out of Play}

\subsection{Law 10 – The Method of Scoring}

\subsection{Law 12 – Fouls and Misconduct}

\subsection{Law 13 – Free Kicks}

\subsection{Law 14 – The Penalty Kick}

\subsection{Law 15 – The Throw-In (Kick-in)}

\subsection{Law 16 – The Goal Kick}

\subsection{Law 17 – The Corner Kick}



BELOW THIS LINE IS THE OLD STRUCTURE

WE MIGHT USE SOME OF IT IN THE FUTURE

====================================
\subsection{Structure of the Game}
\label{sec:game_structure}
\subsection{Game Periods and Robot States}
\label{sec:game_states}
\subsubsection{Game Periods}
\subsubsection{Robot States}

\subsection{Kick-off}
\subsubsection{Field-Side Selection and Initial Kick-off}
\subsubsection{Initial Kick-off}
\label{sec:kick-off}
\subsubsection{Ball in play}
The ball is in play and kick-off ends once:
\begin{itemize}
  \item it is touched by the attacking team and has visibly moved at least one ball radius from its initial position, or
  \item \textit{KickOffBallFreeTime} \todo{Move/define time constants} seconds have elapsed in the \texttt{playing} state.
\end{itemize}
The GameController and head referee will indicate this by the call ``Ball Free''.

\subsubsection{Ball out of play}
The ball is out of play when:
\begin{itemize}
  \item the ball has left the field of play, by wholly crossing the touchline or goal line (see Section~\ref{sec:inside_outside}), or
  \item play has been stopped by the referee.
\end{itemize}
The ball remains in play at all other times.

\subsection{Goals}
\subsubsection{Goal Scored}
\label{sec:goal}

A goal, including an own goal, is scored when the ball is considered \textit{inside} the goal as defined in Section~\ref{sec:inside_outside}.\footnote{
  The goal line is part of the field.
}

The head referee signals a goal by a single whistle blow, followed by the call ``Goal \textless color\textgreater''.
The head referee should point with one arm towards the center of the field.
To assist robots listening for whistles, the referee should blow the whistle from on the carpet at the end of the fields where the goal was scored.

After a team scores a goal, the game proceeds with a kick-off (\cf~\cref{sec:kick-off}) for their opponents.
The GameController signal (to the robots) of a goal being scored, will be delayed by \qty{\GoalScoredDelay}{\second}.

\subsubsection{Invalid Goal}
\subsubsection{Competition Rules}

\subsection{Indirect Kick}

\subsubsection{Fallback mode}


\subsection{Free Kick}
\label{sec:free_kick}

A free kick is a method of restarting play after various stoppages.
Free kicks are classified as either \textbf{direct} or \textbf{indirect} (see~\cref{sec:direct_free_kick,sec:indirect_free_kick}).

\unsure{Discuss: Maintain indirect + direct free kicks, AND indirect kick rule?}

A free kick is initiated in the following situations:
\begin{itemize}
  \item When the ball goes over the touchlines, termed \emph{Kick-in} or \emph{Throw-in} (\cf~\cref{sec:kick_in}) --- indirect free kick.
  \item When the ball goes over the goal line having last touched a player of the defending team, termed \emph{Corner Kick} (\cf~\cref{sec:corner_kick}) --- direct free kick.
  \item When the ball goes over the goal line having last touched a player of the attacking team, termed \emph{Goal Kick} (\cf~\cref{sec:goal_kick}) --- direct free kick.
  \item When a pushing foul or other infringement is awarded (\cf~\cref{sec:player_pushing}), termed a \emph{Pushing Free Kick} --- direct free kick.
\end{itemize}

The head referee will announce a free kick by calling one of:
\begin{enumerate}
  \item ``Kick-in \textless color\textgreater'' / ``Throw-in \textless color\textgreater'' for the team awarded the kick-in.
  \item ``Corner Kick \textless color\textgreater'' for the team awarded the corner kick.
  \item ``Goal Kick \textless color\textgreater'' for the team awarded the goal kick.
  \item ``Foul \textless offending color\textgreater \textless offending number\textgreater'' for pushing free kicks.
\end{enumerate}

The GameController will then activate the sub-state for the respective free kick.
The team awarded the free kick (termed the \emph{attacking team}) has \qty{\FreeKickTime}{\second} to complete the kick.

\subsubsection{Direct Free Kick}
\label{sec:direct_free_kick}

A direct free kick allows the attacking team to score a goal directly from the kick:
\begin{itemize}
  \item If a direct free kick is kicked directly into the opponent's goal by the attacking team, a goal is awarded.
  \item If a direct free kick is kicked directly into the team's own goal, a corner kick is awarded to the opposing team.
\end{itemize}

The following set plays are direct free kicks: \emph{Corner Kick}, \emph{Goal Kick}, and \emph{Pushing Free Kick}.

Note: If the attacking team fails to execute the free kick within the time limit, the defending team may also score directly (see Failed Free Kick in~\cref{sec:free_kick_execution}).

\subsubsection{Indirect Free Kick}
\label{sec:indirect_free_kick}

An indirect free kick requires the ball to be touched by another player before the attacking team can score a goal:
\begin{itemize}
  \item A goal can only be scored by the attacking team if the ball has been touched by another player (of either team) after the kick, before being kicked again into the goal.
  \item If an indirect free kick is kicked directly into the opponent's goal by the attacking team without touching another player, a goal kick is awarded to the opposing team.
  \item If an indirect free kick is kicked directly into the team's own goal, a corner kick is awarded to the opposing team.
\end{itemize}

The following set plays are indirect free kicks: \emph{Kick-in} / \emph{Throw-in}.

Note: If the attacking team fails to execute the free kick within the time limit, the defending team may score directly without the indirect requirement (see Failed Free Kick in~\cref{sec:free_kick_execution}).

\subsubsection{Visual Gesture}
\label{sec:free_kick_gesture}

\todo{Define visual gesture requirements for different divisions}

The referee communicates which team is to take the free kick via a visual gesture.
The gesture consists of raising one arm horizontally and sideways, parallel to the field line, while the other arm remains at rest.
The raised arm points towards the defending team's goal (\ie the goal of the team that is \textit{not} taking the free kick).

This gesture must be held for at least 5 seconds after the GameController signal is sent.
The referee should stand as close as possible to the T-junction joining the halfway line to the touchline opposite to the technical area.

\subsubsection{Execution}
\label{sec:free_kick_execution}

The referee places the ball according to the type of free kick (see~\cref{sec:kick_in,sec:goal_kick,sec:corner_kick} for specific ball placement rules).
For a \emph{Pushing Free Kick}, the ball is left in place, repositioned only if required by the pushing rules (\cf~\cref{sec:player_pushing}).

\paragraph{Avoidance Region}

During a free kick, only the attacking team may approach within the avoidance region of the ball. For a goal kick, the avoidance region is the entire penalty area. For all other free kicks, the avoidance region is division dependent: \FreeKickRadius.
\unsure{Confirm avoidance radius values for each field size or division}

All robots of the defensive team must immediately move away from the ball to outside the avoidance region when the free kick is awarded.
Defensive robots that violate these restrictions are penalized with ``Illegal Positioning'' (\cf~\cref{sec:illegal_positioning}), which results in a standard removal penalty (\cf~\cref{sec:removal_penalty}).
Additional penalties against any further robots during the free kick, including pushing, do not result in an additional free kick but still incur the appropriate removal penalty.

\paragraph{Completion}

A free kick is deemed completed and play returns to normal when:
\begin{itemize}
  \item The attacking team moves the ball clearly (except for a robot getting up, which is exempt from this rule), or
  \item The \qty{\FreeKickTime}{\second} time period expires (or game time expires).
\end{itemize}

The head referee announces completion by calling ``Ball Free'', and the GameController resumes the \texttt{playing} state.

\paragraph{Failed Free Kick}

If the attacking team fails to execute the free kick within the allotted \qty{\FreeKickTime}{\second}, the free kick expires and play resumes normally.
In this situation:
\begin{itemize}
  \item The defending team may play the ball immediately after the free kick expires.
  \item If the defending team plays the ball first (before the originally attacking team), they may score a direct goal regardless of whether the original free kick was direct or indirect.
\end{itemize}

\textbf{Example:} The blue team is awarded a kick-in (indirect free kick) but does not play the ball within \qty{\FreeKickTime}{\second}. The referee calls ``Ball Free''. A red robot immediately kicks the ball directly into the blue goal. The goal counts, as the failed-free-kick exception allows the defending team to score directly.

\subsection{Kick-in / Throw-in}
\label{sec:kick_in}

A kick-in (or throw-in) is an \textbf{indirect free kick} (\cf~\cref{sec:indirect_free_kick}). A kick-in is awarded to the opponents of the player who last touched the ball when the whole of the ball leaves the field of play by crossing the touchline, either on the ground or in the air (see Section~\ref{sec:inside_outside}). Balls are deemed to be out based on the team that last touched the ball, irrespective of who actually kicked the ball.

\subsubsection{Ball Placement}

The ball is placed on the touchline at the point where it left the field.

\subsubsection{Throw-in Option}

Robots may also perform a throw-in with their hands instead of kicking. When performing a throw-in with hands, the robot must:
\begin{itemize}
  \item Face the field of play.
  \item Have part of each foot either on the touchline or on the ground outside the touchline.
  \item Hold the ball with at least one hand.
  \item Deliver the ball from behind and over its head.
  \item Release the ball within 10 seconds.
\end{itemize}

If a robot attempts to perform a throw-in with hands and fails to respect these rules, a free kick is awarded to the opposing team.

\subsubsection{Examples}

\textbf{Example 1:} A blue robot at midfield kicks the ball over the left touchline \qty{2}{\metre} into the red half of the field.
The referee calls ``Kick-in red'' and the ball is replaced on the left touchline where it went out.

\textbf{Example 2:} A blue robot kicks the ball but the ball touches a red robot at midfield before leaving the field near the halfway line.
The ball is regarded as out by red; the referee calls ``Kick-in blue'' and the ball is replaced on the touchline where it went out.

\subsection{Goal Kick}
\label{sec:goal_kick}

A goal kick is a \textbf{direct free kick} (\cf~\cref{sec:direct_free_kick})  awarded when the whole of the ball passes over the goal line, either on the ground or in the air, having last touched a player of the attacking team, and a goal is not scored.

\subsubsection{Ball Placement}
The ball is placed on the corner of the goal area on the same side of the field that the ball went out.
That is, the corner inside the field where the goal area line meets the goal line (not the T-junction on the goal line itself).

\subsubsection{Avoidance Region}
For a goal kick, the avoidance region is the entire penalty area.
All robots of the opposing team must remain outside the penalty area until the ball is in play (\cf~\cref{sec:free_kick_execution}).

\subsubsection{Example}
\textbf{Example:} A blue robot kicks the ball out the end of the field to the right of the goal the red team is defending.
The referee calls ``Goal Kick red'' and the ball is placed on the right corner of the goal area.

\subsection{Corner Kick}
\label{sec:corner_kick}

A corner kick is a \textbf{direct free kick} (\cf~\cref{sec:direct_free_kick}) awarded when the whole of the ball passes over the goal line, either on the ground or in the air, having last touched a player of the defending team, and a goal is not scored.

\subsubsection{Ball Placement}

The ball is placed on the corner of the field on the same side that the ball went out.

\subsubsection{Example}

\textbf{Example:} The red goalkeeper kicks the ball out the end of the field to the right of the goal.
The referee calls ``Corner Kick blue'', the ball is placed on the corner to the right of the goal, and a free kick is started.

\subsection{Penalty Kick}

\subsection{Game Stuck}
\subsubsection{Local Game Stuck}
\subsubsection{Global Game Stuck}

\subsection{Request for Pick-up}

\subsection{Timeout}
\subsubsection{Request for Timeout}
\label{sec:request_for_timeout}

Each team can call a \textbf{maximum of 1 timeout per game} with a total time of no more than \textbf{5 minutes}.
During this time, both teams may change robots, change programs, or anything else that can be done within the time allotted.
During normal game time, a team may call a timeout at any stoppage of play (after a goal, stuck game, before a half, etc.).
Alternatively, a team may call a timeout before a penalty shootout if they have not used their timeout yet (\cf \cref{sec:penalty_shoot-out}).

The timeout ends when the team that called the timeout says they are finished, at which time they must be ready to play.
The other team must be ready to play at the time the timeout runs out, or \textbf{2 minutes} after a prematurely called end of the timeout, whichever is earlier.
If the other team is not ready to play in time, it has to call a timeout of its own.

The clock stops during timeouts, even during the preliminaries, and is reset to the time when the current stoppage of play began.

After the completion of the timeout, the game resumes with a kick-off for the team which did not call the timeout.

If a team is not ready to play at the assigned time for a game, the referee will call the timeout for that team.
After the expiration of such a timeout, if the team is still not ready to play then the referee shall start the game with only one team on the field.
The team that was not ready can return its robots to the field as per the rules for ``Request for Pick-up''.
If both teams are not ready, the referee will call timeouts for both teams.
This ``double timeout'' expires after 10 minutes.

\subsubsection{Referee Timeout}

\subsection{Extra Time}

\subsection{Mercy Rule}
\label{sec:mercy_rule}

A game will conclude once the game score shows a goal difference of 10.
Ending the game is mandatory once a goal difference of 10 is reached.

\subsection{Drop Ball Rule}

\subsection{Ball Stop Rule}

\subsection{Determine the Winner of a Match}
\subsubsection{Winning Team}
\subsubsection{Winner after Drawn}

\subsection{Penalty Kick Shoot-Out}
\subsubsection{Penalty Kick}
\subsubsection{Sudden Death Shoot-Out}
