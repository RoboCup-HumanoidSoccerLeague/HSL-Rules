% !TeX root = ../Rules.tex
% !TeX spellcheck = en_US
\section{Game Process}
\label{sec:game}



Law 7 and the following ones.
\subsection{Law 7 – The Duration of the Match}
% Following subsection was edited by Prof. Azer Babaev 
\subsection{Law 8 – The Start and Restart of Play}
\label{sec:Law 8 – The Start and Restart of Play}

A kick-off starts both halves of a match, both halves of extra time and restarts play after a goal has been scored. Free kicks (direct or indirect), penalty kicks,
throw-ins, a goal kicks and corner kicks are other restarts (see Laws 13–17).
A dropped ball is the restart when the referee stops play and the Law does not
require one of the above restarts.

If an offence occurs when the ball is not in play, this does not change how
play is restarted.

\subsubsection{Kick-off}
\textbf{Procedure}
\begin{itemize}
    \item The referee tosses a coin and the team that wins the toss decides which goal to attack in the first half or to take the kick-off.

    \item Depending on the above, their opponents take the kick-off or decide which goal to attack in the first half.

    \item The team that decided which goal to attack in the first half takes the kick-off to start the second half.

    \item For the second half, the teams change ends and attack the opposite goals. The kick-off at the beginning of the second half has to be taken by the team which did not take the kick-off in the beginning of the first half. 

    \item After a team scores a goal, a kick-off has to be taken by their opponents.
\end{itemize}

For every kick-off:
\begin{itemize}
    \item all players must be in their own half of the field of play,

    \item the opponents of the team taking the kick-off must be outside from the center circle until ball is in play,

    \item the ball must be stationary on the center mark,

    \item the referee gives a signal,

    \item the ball is in play when it is kicked and clearly moves,

    \item if the ball does not move 10 seconds after the referee has given the signal then the ball is in play and the opponents of the team taking the kick-off may enter into the center circle, \info{From humanoid league, to be discussed}

    \item a goal may not be scored directly against the opponents from the kick-off,

    \item if the team taking the kick-off has three or more robots on the field then two different robots need to touch the ball before scoring a goal. If a team taking the kick-off has only two or less robots on the field, the robot taking the kick-off has to touch the ball at least one time outside the center circle before scoring a goal.\info{This statement needs confirmation by TC} 
\end{itemize}
\textbf{Offences and sanctions}

In the event of any kick-off procedure offence, the kick-off is retaken.

\subsubsection{Dropped ball}\info{Dropped ball definition and procedure are different in FIFA rules and in RC-HL rules. RC-HL rules are used in current edition because FIFA rules of dropped ball are hardly to be implemented by robots}
\textbf{Definition of dropped ball}

A dropped ball is a method of restarting play when, while the ball is still in play, the referee is required to stop play
temporarily for any reason not mentioned elsewhere in the Laws of the Game. In the virtual competition, the only reason for a dropped ball to be called is that the ball has moved less than 5 centimeters in the last 2 minutes of play.

\textbf{Procedure}

The game is continued at the center mark. A goal can be scored directly from a dropped ball. The procedure for dropped ball is the same as for kick-off, except that the players of both teams must be outside the center circle. The ball is in play immediately after the referee gives the signal

\textbf{Offences and sanctions}
\begin{itemize}
    \item If a player moves too close to the ball before the referee gives the signal, a kick-off is awarded to the opponent team.

    \item The ball is dropped again:
        \begin{itemize}
            \item if it is touched by a player before it makes contact with the ground,
            \item if the ball leaves the field of play after it makes contact with the ground, without a player touching it.
        \end{itemize}
\end{itemize}
% End of subsection which was edited by Prof. Azer Babaev

\subsection{Law 9 – The Ball In and Out of Play}

\subsection{Law 10 – The Method of Scoring}

\subsection{Law 12 – Fouls and Misconduct}

\subsection{Law 13 – Free Kicks}

\subsection{Law 14 – The Penalty Kick}

\subsection{Law 15 – The Throw-In (Kick-in)}

\subsection{Law 16 – The Goal Kick}

\subsection{Law 17 – The Corner Kick}



BELOW THIS LINE IS THE OLD STRUCTURE

WE MIGHT USE SOME OF IT IN THE FUTURE

====================================
\subsection{Structure of the Game}
\label{sec:game_structure}
\subsection{Game Periods and Robot States}
\label{sec:game_states}
\subsubsection{Game Periods}
\subsubsection{Robot States}

\subsection{Kick-off}
\subsubsection{Field-Side Selection and Initial Kick-off}
\subsubsection{Initial Kick-off}
\subsubsection{Kick-off}
\label{sec:kick-off}
\subsubsection{Ball in play}

\subsection{Goals}
\subsubsection{Goal Scored}
\label{sec:goal}

A goal, including own goal, is achieved when the entire ball (not only the center of the ball) goes over the goal-side edge of the goal line, \ie the ball is completely inside the goal.\footnote{
  The goal line is part of the field.
}

The head referee signals a goal by a single whistle blow, followed by the call ``Goal \textless color\textgreater''.
The head referee should point with one arm towards the center of the field.
To assist robots listening for whistles, the referee should blow the whistle from on the carpet at the end of the fields where the goal was scored.

After a team scores a goal, the game proceeds with a kick-off (\cf~\cref{sec:kick-off}) for their opponents.
The GameController signal (to the robots) of a goal being scored, will be delayed by \qty{\GoalScoredDelay}{\second}.

\subsubsection{Invalid Goal}
\subsubsection{Competition Rules}

\subsection{Kick-in / Throw-in}

\subsection{Goal-Kick}

\subsection{Corner-Kick}

\subsection{Free Kick}
\subsubsection{Direct Free Kick}
\subsubsection{Indirect Free Kick}
\subsubsection{Visual gesture}
\subsubsection{Execution}

\subsection{Indirect Kick}
\subsubsection{Fallback mode}

\subsection{Penalty Kick}

\subsection{Game Stuck}
\subsubsection{Local Game Stuck}
\subsubsection{Global Game Stuck}

\subsection{Request for Pick-up}

\subsection{Timeout}
\subsubsection{Request for Timeout}
\label{sec:request_for_timeout}

Each team can call a \textbf{maximum of 1 timeout per game} with a total time of no more than \textbf{5 minutes}.
During this time, both teams may change robots, change programs, or anything else that can be done within the time allotted.
During normal game time, a team may call a timeout at any stoppage of play (after a goal, stuck game, before a half, etc.).
Alternatively, a team may call a timeout before a penalty shootout if they have not used their timeout yet (\cf \cref{sec:penalty_shoot-out}).

The timeout ends when the team that called the timeout says they are finished, at which time they must be ready to play.
The other team must be ready to play at the time the timeout runs out, or \textbf{2 minutes} after a prematurely called end of the timeout, whichever is earlier.
If the other team is not ready to play in time, it has to call a timeout of its own.

The clock stops during timeouts, even during the preliminaries, and is reset to the time when the current stoppage of play began.

After the completion of the timeout, the game resumes with a kick-off for the team which did not call the timeout.

If a team is not ready to play at the assigned time for a game, the referee will call the timeout for that team.
After the expiration of such a timeout, if the team is still not ready to play then the referee shall start the game with only one team on the field.
The team that was not ready can return its robots to the field as per the rules for ``Request for Pick-up''.
If both teams are not ready, the referee will call timeouts for both teams.
This ``double timeout'' expires after 10 minutes.

\subsubsection{Referee Timeout}

\subsection{Extra Time}

\subsection{Mercy Rule}
\label{sec:mercy_rule}

A game will conclude once the game score shows a goal difference of 10.
Ending the game is mandatory once a goal difference of 10 is reached.

\subsection{Drop Ball Rule}

\subsection{Ball Stop Rule}

\subsection{Determine the Winner of a Match}
\subsubsection{Winning Team}
\subsubsection{Winner after Drawn}

\subsection{Penalty Kick Shoot-Out}
\subsubsection{Penalty Kick}
\subsubsection{Sudden Death Shoot-Out}
