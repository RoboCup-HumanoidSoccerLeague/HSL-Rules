% !TeX root = ../Rules.tex
% !TeX spellcheck = en_US
\section{Judgement}
\label{sec:judgement}
\info{Move to: Law 5 - The Referee}

The referees are the only humans permitted on the carpeted area (\ie the field and the border area). They have the duty to enforce the rules of play and ensure the game proceeds smoothly before, during, and after the scheduled time.

\subsection{Head Referee}
\info{Law 5.1 - The authority of the referee}
\label{sec:head_referee}
The head referee is the principal authority of the game and takes all decisions regarding the flow of the match and the enforcement of the rules. 
\info{Law 5.2 - Decisions of the referee}
Any decision of the head referee is valid. The head referee’s decision is final and can not be changed afterwards, even by video proof. There is no discussion about decisions during the game, neither between the assistant referees and the head referee, nor between the audience or the teams and the head referee.

\textbf{Calls.} The head referee announces decisions by a clear loud call, and (as required) whistle sound. The whistle, or where there is no whistle the first verbal word of the referees calls, defines the point in time at which the decision is made. The referees should make efforts to use consistent and clear calls, and it is preferable for referees to use the calls as specified in these rules.\footnote{The calls specified in these rules are detailed in English. With the agreement of the teams, the referees may use suitable calls in any language. The exception to this are technical challenges that depend on the calls as specified.} The intention of specifying the referee calls is for clarity and consistency across games.

\textbf{Whistle.} Where a whistle is required, the head referee first whistles and then announces the reason for the whistle. The head referee may choose to use any normal sports whistle. Each whistle sound should be short and not too loud as to interfere with other fields and simultaneous games. The head referee must only sound the whistle in circumstances described in these rules.

The head referee should avoid handling the ball (except for placing a ball for kick-off), and avoid handling the robots. Their duty is to monitor and adjudicate the game. The head referee should only handle robots and the ball if absolutely necessary to expedite gameplay or removal of penalized robots, where the assistant referees are otherwise occupied or too far away, and only if it doesn't compromise their own safety.

\info{Law 5.5 - Referee's equipment}
The head referee should be equipped with a suitable referee jersey, whistle, coin, and black or dark-blue socks.

\info{Law 5.3 - Powers and duties - missing larger parts from Fifa rules}
The head referee may decide at any point before or during a game to relocate any objects around the field, or direct persons to another position around the field.

\subsection{Assistant Referees}
\label{sec:assistant_referee}
\info{Law 6 - The Other Match Officials}

\textit{The following paragraph depends on future safety decisions, specifically regarding whether assistant referees will be allowed to handle the robots, and in which circumstances. Such decisions may also depend on the division.}

\info{Law 6.1 - Assistant referees}
The assistant referees handle the \improvement{Potentially add: (unless otherwise specified by division rules)} robots and the ball. They take the robots out when they are penalized, and they put the robots in again. If a team requests to pick up a robot, an assistant referee will pick it up and give it to one of the team members once the head referee approves. An assistant referee will also put the robot back on the field. An assistant referee will also replace the ball when it goes off the field or becomes stuck between a players feet. At the discretion of the head referee, more tasks can be delegated to the assistants (Section 5.4).

Assistant referees should only enter the field to execute a decision made by the main referee. They should not prevent robots from falling during the game.

A game has at least two assistant referees. If agreed upon by the referee teams or under certain circumstances, additional assistants may be present, up to a total maximum of four. See \autoref{sec:referee_duty}.

The assistant referees should be equipped with a suitable referee jersey, and black or dark-blue socks.

\subsubsection{The Stationary Assistant}

\textit{This only makes sense to have if the indirect kick rule and/or the referee gestures from SPL-2025 are used.}

\subsection{GameController Operator}
\label{sec:gamecontroller_operator}

The operator of the GameController sits at a PC in the technical area. As with the head referee, the operator should make efforts to use consistent and clear calls.

They will signal any change in the game state or penalties to the robots via the wireless as they are announced by the head referee. Note that for both kick-offs and goals, the moment of whistling is determining, not the verbal announcement of the head referee. They should repeat the call of the head referee as they do so, to make sure it was heard correctly.

The operator will also inform the assistant referees when a timed penalty is over and a robot has to be placed back on the field. \improvement{If the penalty procedure would benefit from an advance warning about 10 seconds before the end of the penalty, it can be mentioned here.} They should announce events that occur automatically in the GameController due to elapsed time, such as the ball coming into play after a kick-off, penalty kick or free kick, or the state changing from ready to set.

They are also responsible for keeping the time of each half. They should count aloud the remaining seconds in a half once the time remaining is 5 s or less.

\subsection{Game Process}

\subsubsection{Pre-game Referee Meeting and Task Delegation}
\label{sec:referee_delegation}
\info{Law 5.x}

Before the game starts (no later than 10 minutes before), the people scheduled to serve as referees meet up to discuss the upcoming game. At least, they must decide which team is going to provide the head referee and the GameController operator and which is going to provide the assistants, and whether the head referee is going to delegate any duties to the assistants. Other topics that ensure a smooth cooperation among the referees can also be discussed.

The head referee should talk to the assistants to determine what tasks or lesser decisions, if any, they wish to delegate to them to ensure that the game is arbitrated as smoothly as possible. This is left to the discretion of the head referee, based on their expertise in the role and their ability to focus on multiple events happening in the game at the same time and apply the corresponding rules.

The head referee must clearly communicate what tasks are delegated to which person, so that everyone understands their duties during the game.

If no agreement can be found, the default is that the responsibility for most calls and decisions falls upon the head referee, as determined by the rules.

Common examples of tasks that can be delegated are:

\begin{itemize}
    \item Determining which team is to be given a free kick when the ball goes out of the field (Section ???) and communicating this to the head referee. The final call is still made by the latter.
    \item Indicating violations requiring a penalty, with the final call still being made by the head referee.
    \item \textit{More examples should be added depending on how the rules are shaped up}
\end{itemize}

The above list is not prescriptive: the head referee can always choose to delegate zero, one, some, or all of these tasks. It is also not exhaustive: through discussion among the head referee, the assistants, and the GameController operator, other tasks not listed here may be identified and delegated.

Care should be taken to not overburden any one person and not to blur the roles of head and assistant referees. Conflicts in the authority of the referees should be avoided, but if any do occur, the head referee’s decision is final.

\subsubsection{Pre-game Team Meeting}
\info{Law 7.x}


Both teams send a representative called team captain to the field 10 min before their match starts. This time should be used to welcome each other, assign team colors (see Section ???), choose side and kick-off (see Section ???), and discuss any other topics related to the match.

\subsubsection{Referee--Team Communication}
\label{sec:referee_team_communication}
\info{Law 5.x}

During the match, only the team captains are allowed to communicate with the head referee. Only the team captains and two more people per team are allowed to stay next to the \improvement{The exact name of this area depends on how the tables are laid out} game controller tables. The rest of the team locates themselves around the other sides of the field if they want to watch the match. This allows the referees easier communication with the team and the game controller operator gets less disturbed.

After the match the teams thank the referees for their duty.

During all phases of the match teams and referees are communicating with respect to each other.

\subsubsection{Referees during the Match}
\label{sec:referee_during_match}
\info{Law 5.5 - Referee's equipment}


The head referee and the assistant referees should wear socks of black or dark blue color and avoid reserved colors (white and green) in their leg clothing. They may \improvement{Pending safety for bigger divisions} enter the field in particular situations, e. g., to remove a robot when applying a penalty. They should avoid interfering with the robots as much as possible.

\subsubsection{Visual Signal}
\label{sec:referee_visual_signal}
\info{Law 5.6 - Referee signals or Law 6.6 Assistant referee signals}


\textit{Pending decision}

\subsubsection{Additional Assistant Referee}
\label{sec:additional_assistant_referee}
\info{Law 6.3 - Additional assistant referees}

\subsection{Referee List for Friendly Games}
\label{sec:referee_list}
\info{Law 5.x}

During a competition, especially (but not only) during the setup days, several teams may want to participate in friendly games with each other if a field is available to play in. People willing to volunteer for judging these games as head referee, assistant referee or GameController operator may submit their name to a list managed by the Organizing Committee, so that the teams organizing the friendly game are aware of their availability. This is especially recommended for those who wish to gain referee experience.

Ultimately, the teams organizing the friendly game are still free to decide whether to call volunteers from the list or otherwise choose their referees.

The Organizing Committee is in charge of maintaining the referee list and should be approached at the competition site if one should want to volunteer.

