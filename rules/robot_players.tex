% !TeX root = ../Rules.tex
% !TeX spellcheck = en_US
\section{Teams and Players}
\label{sec:teams_and_players}

\subsection{Divisions and Configurations}
\label{sec:divisions}

A team belongs to one division, which is one of Small, Middle and Large. Main competition matches only take place between members in the same division.

Teams in each division can choose to play in two team configurations: Foundation - with a smaller number of players; and Advanced - with a larger number of players. If in a game one of the teams chooses to play in a Foundation configuration, then both teams must play in this configuration to ensure fairness. The maximum number of players in a game for each configuration is shown in Table \ref{tab:number_of_players}.

In the Middle division, starting from the quarter-finals onwards, all matches will be played in the Advanced configuration.

Both the division and the configuration of a team are defined before the competition (precise modality TBD) and cannot change during the competition itself.

\improvement{Maybe move some definitions from the beginning of Comp. Structure over here?}

\subsection{Number of Players}
\label{sec:number_of_players}
A match is played by two teams, each with a \textbf{maximum} number of players determined by the division the game takes place in and the configuration level of the playing teams, as illustrated by the following table:

\begin{tblr}{
  width=1.0\linewidth,
  colspec = {Q[l]Q[l]X[l]},
  row{1}={fg=black, font=\bfseries}, %, 
  hline{2} = {solid},
  hline{3-Y} = {dashed},
  vline{2,3} = {abovepos = -1, belowpos = -1},
}
Division & Foundation team & Advanced teams\\
%
Small & 4 & 7 \\
Mid   & 3 & 5 \\
Large & 3 & 5 \\
\label{tab:number_of_players}
\end{tblr}




At most one player per team on the field may be designated as \emph{goalkeeper}, the others are all \emph{field players}.

When playing at full strength, a team must have a \emph{goalkeeper} on the field. \improvement{Goalie distinguishing feature TBD: Number? Color?}

Each of the players has a unique jersey number from the set $\{1, 2, 3, \ldots, 20\}$. \improvement{Will we have flexible numbers on the game controller? Otherwise, the numbers should be limited to the game controller numbers.}

\subsection{Number of Substitutes}
\label{sec:number_of_substitutes}
In addition, each team may prepare \emph{substitute players} \improvement{Number of allowed substitutes, needs clarity - limited, unlimited?} outside the field.
A \emph{substitute player} may be substituted in to become a \emph{field player} or \emph{goalkeeper}.

\subsection{Substitution procedure}

\subsubsection{Changing the goalkeeper}
Any of the other players may change places with the goalkeeper, provided that the referee is informed before the change is made. The change is requested during a stoppage in the match.

All substitutes are subject to the authority and jurisdiction of the referee, whether called upon to play or not.

\improvement{Some more sections of the Humanoid rule book, like substitution procedures for players and goalkeeper, sanctions, etc. should be moved to the chapter Game Process }



\section{Robot Players}
\label{sec:robot_players}

\subsection{The Design of the Robots}
\label{sec:design_of_robots}
Robots participating in the \leaguenameabbr must have a human-like body shape with a torso, head, two arms, and two legs, as well as human-like symmetry and proportions regarding sizes of the body parts and weight distribution. \info{Definition of humanoid robot from Kajima et al, 2005}

The robots must be able to stand upright on their feet, to walk on their legs and to be able to recover from a fall (get back to a standing position). 

The only allowed modes of locomotion are bipedal walking, running, and jumping. Soccer related movements such as dribbling, kicking, or other forms of ball handling are also allowed.

The design of the robot's arms, including their length and placement, shall permit arm use and behaviors that are reasonably comparable to those of humans. Examples of permitted uses include assisting in getting up after a fall or picking up and throwing the ball (where otherwise allowed by the rules).

Arm configurations that enable behaviors significantly different from those of humans are not permitted. In particular, 
robots must not use their arms to provide continuous support for locomotion, such as walking on arms or using arms as additional legs.

\subsubsection{Size Restrictions}

All robots participating in the \leaguenameabbr must comply with the following restrictions:\improvement{to be discussed - including height restrictions}

The length of the legs $H_{leg}$, including the feet, satisfies 0.35{\textperiodcentered}$H_{top}$ ${\leq}$ $H_{leg}$ ${\leq}$ 0.7{\textperiodcentered}$H_{top}$,  where $H_{top}$ is the height of the top of the robot. 
The length of the leg is measured from the first rotating joint where its axis lies in the plane parallel to the standing ground to the tip of the foot.

A classic piece of human anatomy and art history, Leonardo da Vinci’s “Vitruvian Man” famously depicts a man whose arm span is equal to his height, creating a 1:1 ratio. 
Therefore, the arm span, $A_{span}$, including the hands, should satisfy 0.8{\textperiodcentered}$H_{top}$ ${\leq}$ $A_{span}$ ${\leq}$ 1.2{\textperiodcentered}$H_{top}$.

Based on $H_{top}$, the following size restrictions apply for each division:
%
\begin{itemize}
\item Small:  $H_{top}$ ${\leq}$ 1.1 meters;
\item Middle: $H_{top}$ ${\leq}$ 1.25 meters;
\item Large:  $H_{top}$ ${\leq}$ 1.9 meters.
\end{itemize}

$H_{top}$ is defined as the height of the robot when standing upright (with fully extended knees).
$H_{top}$ is measured with the head of the robot oriented in such a way that it
is tilted to either its maximum upwards tilt angle or the horizon line,
whichever is lower.

The height of the head $H_{head}$, including the neck, satisfies $0.1 \cdot H_{top} \leq H_{head} \leq 0.3 \cdot H_{top}$. $H_{head}$ is defined as the vertical distance from the axis of the first arm joint at the shoulder to the top of the head.

\subsubsection{Weight Restrictions}

The robot's Body-Mass Index (BMI) is defined as follows:
$\mathrm{BMI} = \frac{M}{{H_{top}}^2}$,
where $M$ is the mass of the robot in kg and $H_{top}$ its height in meters.

The Body Mass Index (BMI) of the robot should be: $5 \leq \mathrm{BMI} \leq 30$.

The following weight restrictions apply for each division:
%
\begin{itemize}
\item Small:  Weight${\leq}$ 15 kilograms;
\item Middle: Weight ${\leq}$ 25 kilograms;
\item Large:  Weight ${\leq}$ 80 kilograms.
\end{itemize}

\subsubsection{Safety}\info{From the humanoid league rules, to be discussed}

A player must not use equipment or wear anything that is dangerous to himself or another player.

Robots competing in the physical competition must be equipped with an emergency stop button that makes the robot immediately desist with all motions, 
or ideally go limp and/or cut power to the actuators.
In addition to the emergency stop button, robots may only have up to two additional physical or virtual buttons\info{to be discussed: 4.7.1 allows buttons, community also requests to allow more buttons (NAO). Usage during the game can be restricted}: One to start the robot behaviour and one to stop the behaviour. 
The buttons must be clearly labeled. If the robot has more buttons that cannot be detached, they must be visibly masked during the games.

In ...tbd... size, robot handlers are allowed to carry an additional remote emergency stop button. \improvement{to be discussed, will there still be robot handlers?}
This button must be worn either around the neck or on the belt of the robot handler and must be clearly marked. 
Each emergency stop button can only be connected to the robot of the robot handler that holds the button.
The remote emergency button cannot perform any additional functions and does not replace the regular emergency button. 
Robot handlers must keep their hands clearly away from the button unless the button is being pressed. 
Robot handlers must not use the remote emergency button to intentionally incapacitate their robots.

\improvement{to be discussed, add other safety measures from our discussion?}

\label{sec:design_of_robots}

\subsection{Approved Standard Platforms}
The following commercially available robotic platforms are approved for participation in the \leaguenameabbr. 

The following are the list of pre-approved standard platforms that require no modifications:

\begin{tblr}{
  width=1.0\linewidth,
  colspec = {Q[l]Q[l]X[l]},
  row{1}={fg=black, font=\bfseries}, %, 
  hline{2} = {solid},
  hline{3-Y} = {dashed},
  vline{2,3} = {abovepos = -1, belowpos = -1},
}
Manufacturer & Model & Restrictions\\
%
Aldebaran & Nao V5, and Nao V6 \\
Robotis   & DARwIn-OP \\
Booster   & T1, and K1 \\
Fourier   & GR1, GR1-Pro, and GR2 \\
Unitiree  & G1++, H1++ \\
\end{tblr}

Additional humanoid robot platforms can be approved upon request. Please send your request for approval to \url{rc-spl-tc@lists.robocup.org}. \improvement{Change to new mailing list when ready}

\subsection{Hardware}
\label{sec:hardware}

\improvement{to be discussed, divisions, list of permitted robots, self-built robots, etc.}

Modifications or additions to the robot hardware are allowed.\improvement{to be discussed}


No additional hardware is permitted, including off-board sensing or processing systems.
Additional sensors besides those originally installed on the robots are likewise not allowed.

\subsection{Sensors}
\label{sec:sensors}

Teams participating in the \leaguenameabbr competitions are encouraged to equip their robots with sensors that have an equivalent in human senses. 
These sensors must be placed at a position roughly equivalent to the location of the human's biological sensors. In particular, ...
\improvement{to be discussed}

The sensors and their placement shall be chosen such that they allow the robot a spatial perception similar to humans.
%
The sensors are evaluated along the following two general guidelines:
\begin{enumerate}
    \item \emph{Foster and encourage research and development towards human-like perception capabilities.}\\
    Sensors aiming to directly emulate human senses, like a camera or a microphone. Use of such sensors is explicitly \emph{encouraged}.
    \item \emph{Enable research and development under the constraints of the current state of the art in technology and research, as long as this does not undermine current research efforts as declared in point 1.} \\
    Sensors that compensate for current shortcomings in technology and state of research, like one-dimensional distance sensors with limited range, or two vertically arranged cameras in the robot NAO.
\end{enumerate}

Generally, \emph{passive sensors} are preferred to \emph{active} sensors that actively emit signals.

\subsubsection{Intrinsic Perception and Proprioception}
Any sensing capability aiming at measuring the internal state of the robot is permitted. 
This includes temperature, current consumed by the motors, joint positions, etc.

\subsubsection{Visual Perception}
The visual sensors, e.g., cameras of the robot shall be arranged such that the combined visual field is \emph{contiguous} and limited to dimensions similar to a human, which corresponds to limitations in the opening angle: horizontal $\leq$ \SI{180}{\degree} and vertical $\leq$ \SI{140}{\degree}.

% Placement
The visual sensors shall be located in the head of the robot.

% 
The combined dynamic visual field that can be observed by the robot solely by moving its cameras (similar to human eye movements) and the head is limited to horizontal $\leq$ \SI{220}{\degree} and vertical $\leq$ \SI{180}{\degree}.

The cameras are restricted to visual information in the range of the light visible to humans.

The cameras can provide dense visual information, such as a rasterized image, or sparse information, such as visual events (event cameras).


\subsubsection{Visual (dense) Depth Perception}

Passive integrated devices that provide dense depth information, such as stereo cameras, are permitted and encouraged.

Active integrated devices that provide depth information, such as cameras with an active infrared projector or time-of-flight cameras, are permitted but discouraged.
Their use might be prohibited in the future.


\subsubsection{Orientation}

Sensors providing information regarding the robot's orientation in space with relation to the ground are permitted. This includes sensors such as gyrometer, accelerometer, as well as integrated inertial measurement units (IMU), as long as they provide only relative measurements and no measurements of the absolute direction, such as a compass.

An IMU with an integrated compass can be used as long as the compass is disabled.

The sensors can be placed in the head and/or in the torso of the robot.

%Removed at the Town Meeting
%The number of the orientation sensors is limited to 4.


\subsubsection{Sound Perception}
The sound sensors, e.g., microphones, shall be placed in the head of the robot.
The number of microphones is limited to ??

\subsubsection{Haptic Sensing}
Any passive sensor allowing haptic measurement is permitted, such as force sensors, touch sensors, buttons/bumpers, capacitative touch sensors.

Haptic sensors can be placed at any location of the robot's body and are not limited in number.


\subsubsection{Distance Sensing}

Active and passive sensors for one-dimensional distance sensing with a limited range are permitted if their use is limited to compensate for shortcomings in the spatial awareness of the robot in the close proximity.

%The range is limited to ...
The number of such distance sensors is limited to 4.

The use of distance sensors is discouraged and might be prohibited in the future.

\subsubsection{Summary}
The following table briefly summarizes pre-approved sensors.
The use of listed sensors that are considered human-like is encouraged.
The use of listed sensors that are not considered human-like is accepted, but discouraged and might be prohibited or limited in the future.

\begin{tblr}{
  width=1.0\linewidth,
  colspec = {Q[l]Q[l]X[l]},
  row{1}={fg=black, font=\bfseries}, %, 
  hline{2} = {solid},
  hline{3-Y} = {dashed},
  vline{2,3} = {abovepos = -1, belowpos = -1},
}
Sensor Type & Human-Like & Comments \\

%Visual Perception
RGB camera & yes & opening angle limit: horizontal $\leq$ \SI{180}{\degree}, vertical $\leq$ \SI{140}{\degree}\\
Stereo camera & yes & \\
Event camera & yes & \\

Active RGB-D based on infrared projection & no & discouraged\\
Active RGB-D based on time of flight (TOF) & no & discouraged\\

% sound perception
Microphone array & yes & \\

% Orientation
Gyro & yes & \\
Accelerometer & yes & \\
Compass & no & magnetic measurement sensors are not allowed\\ 
Integrated inertial measurement units (IMU) & yes & no compass or must be disabled and not used\\

% Haptic sensing
Force sensors & yes & \\
Touch sensors & yes & \\
Buttons / Bumpers & yes & \\
Capacitive touch sensors & yes & \\

% Distance sensing
Near range infrared sensor & no & \\
Sonar / ultrasound sensors & no & \\
1D LIDAR and laser range sensor & no & with limited range \\

% Intrinsic sensors
Intrinsic sensors  & yes & temperature, current, joint positions, etc. \\
\end{tblr}

\info{Reinaldo: I think we forgot about compass, so I added to the table}

\unsure{Clarity needed for manufacturers that do not provide IMU details, just access to data}

\subsection{Team Markers}
\label{sec:team_markers}

Robot bodies must be colored mostly with a neutral color, such as black, gray, white, or silver, and be non-reflective.

Robots must use colored jersey shirts as team markers. A jersey is a tank top-style shirt that may start at the neck and go down to the waistline, and does not cover the arms of the robots. It must cover at least 50\% of the upper body of the robot.  

Jerseys must have a primary color that comprises more than half of the jersey. The primary color must be recognizable from the front and back.

Jerseys should not contain distracting patterns that could be confused with other elements of the soccer field, such as field lines or the ball.

Jersey material must be non-reflecting, non-shiny, and non-glittery. Jerseys may be manufactured from fabric and fine mesh.

The two teams must wear colors that distinguish them from each other and also the referee and the assistant referees.

Each team must provide two jersey designs in distinctly different solid colors. The jersey color must clearly contrast with any uncovered parts of the robot to ensure reliable identification.

Robots must display clear identification numbers (player number): a large number on the back and a smaller number on the front.

The goal keeper robot must be marked uniquely so that it can be easily distinguished from the other robots of a team by the referees.

Sponsorship logos will be allowed on jerseys, though detailed guidelines are still being finalized.

Jersey designs must be submitted for approval prior to the competition.

\info{to be discussed}

\subsection{Goalkeeper}
\label{sec:goalkeeper}

The \emph{goalkeeper} may use any of the allowed jersey numbers.
The \emph{goalkeeper} must wear a jersey with a primary color different from the primary colors used by the \emph{field players} of both teams.

\subsection{Communication and Control}

Robots participating in the \leaguenameabbr competitions must act autonomously while a competition is running. No external power supply, teleoperation, remote control, or remote brain of any kind is allowed.
Communication is only allowed among robots on the field, \improvement{between the robots and the referees,} and between the robots and the GameController.

\subsubsection{Non-wireless Communications}
\label{sec:acoustic}

In general, there are no restrictions on communication between robots in play on the field using visual signaling (\eg, gestures) or the robot's built-in microphones, speakers, and infrared transceivers.
However, communication that causes excessive discomfort to an audience, affects the safety of an audience, or violates normal playing rules is not permitted.

\subsubsection{Wireless Communications}
\label{sec:wireless}

... \improvement{To be discussed. Take this section from the SPL rules?}

The use of remote processing/sensing is prohibited.