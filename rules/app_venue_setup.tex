% !TeX root = ../Rules.tex
% !TeX spellcheck = en_US
\section{Venue Setup}
\label{sec:venue_setup}

This appendix contains relevant information for setting up and running RoboCup competitions. It is intended to serve as a guideline for local organisers on how to deal with infrastructure, playing fields and other relevant issues.

\subsection{Field Construction}
The standard soccer field consists of 8mm artificial turf mounted on a flat wooden base. The dimensions of the different field sizes are shown in table \ref{tab:field_dim}. 
More detailed technical drawings are provided in Appendix \ref{sec:technical_drawings} to this document.
Note that the penalty mark is a cross\info{tbd - keep the SPL cross or make it a circle?} and there is a dash at the midpoint of the halfway line.
White field lines can be made of the same 8mm artificial turf, but in white (\eg made of white artificial turf), spray-painted or taped. 
Regardless of the solution, the field lines must be durable throughout the competition.
Care should be taken to ensure the fields are as flat and level as possible.
Additionally, the wooden base should be well-supported and should not give when humans stand, walk or play soccer on it.

\info{Boards and nets around the fields are required given the current developments}
\info{Secure placement of cameras on the side and/or the corners of the field }

\subsection{Lighting Conditions}
It is expected that the venue provides reasonable lighting suitable for general visibility (\eg indoor with artificial lighting, outdoor with natural lighting, or a combination of both).
The lighting conditions depend on the actual venue. 
Fields should be placed near or under windows where possible. 
Whether window lighting is used or not, ceiling lights should be provided as necessary so that most of the field is at least \qty{300}{\lux} (preferably \qty{400}{\lux}).
This lighting may include variations such as glare, brightness, shadows, or mixed lighting conditions that can change throughout the match.
However, the lighting must be predominantly white, and colored lighting that significantly changes the perceived color of the field or ball is not allowed.
Natural and non-natural light must be free to reach the field.
The technical committee can delimit a zone near the field where humans must not stand and where any items blocking the light sources are forbidden.

\subsection{Field Placement}
Fields may be located close to one another. 
Barriers (i. e., boards and nets) must be placed around every playing field to stop shots that could otherwise fly into neighbouring fields or into the audience.

\info{More details incl. table placement needed}

\subsection{Technical Area}
Next to the field, along one of its longer sides, there is the technical area. It must contain a computer
with two monitors for the purpose of sending GameController messages to the robots and observing
compliance with the wireless network usage rules. In addition, there can be a larger monitor facing
the field showing the Game State Visualizer.
The technical area can also contain cameras directed at the field and related equipment with the
following use cases:
\begin{itemize}
    \item Stream video data to the Internet to allow the public to watch games
    \item Record video data for later analysis and creation of statistics
\end{itemize}
The cameras must be protected in such a way that deflected balls cannot hit them.


